%\documentclass{beamer}
\usepackage[T1]{fontenc}
\usepackage{pifont}
\usepackage{threeparttable}
\usepackage{subcaption}
\usepackage{tikz-qtree}
\usepackage{listings}
\usepackage[american]{babel}
\usepackage{csquotes}
\usepackage[style=apa, backend=biber]{biblatex}
\usepackage{tikz}
\usepackage{multicol}
\usepackage{booktabs}
\usepackage{graphicx}
\usepackage{neuralnetwork}
\usepackage{hyperref}

\usepackage{minted}
\definecolor{listingbg}{rgb}{0.87,0.93,1}
\setminted[python]{
breaklines,
linenos,
fontsize=\scriptsize,
frame=single,
xleftmargin=0pt}

\hypersetup{
    pdfborder={0 0 0},
    colorlinks=true,
}
\usetheme[block=fill,subsectionpage=progressbar,sectionpage=progressbar]{metropolis} 

\definecolor{Purple}{HTML}{911146}
\definecolor{Orange}{HTML}{CF4A30}

\setbeamercolor{alerted text}{fg=Orange}
\setbeamercolor{frametitle}{bg=Purple}

\setbeamercovered{still covered={\opaqueness<1->{5}},again covered={\opaqueness<1->{100}}}

\lstset{
    basicstyle=\scriptsize\ttfamily,
    columns=flexible,
    breaklines=true,
    numbers=left,
    %stepsize=1,
    numberstyle=\tiny,
    backgroundcolor=\color[rgb]{0.85,0.90,1}
}

\lstnewenvironment{lstlistingoutput}{\lstset{
        basicstyle=\footnotesize\ttfamily,
        columns=flexible,
        breaklines=true,
        numbers=left,
        %stepsize=1,
        numberstyle=\tiny,
        backgroundcolor=\color[rgb]{.7,.7,.7}}}{}


\lstnewenvironment{lstlistingoutputtiny}{\lstset{
        basicstyle=\tiny\ttfamily,
        columns=flexible,
        breaklines=true,
        numbers=left,
        %stepsize=1,
        numberstyle=\tiny,
        backgroundcolor=\color[rgb]{.7,.7,.7}}}{}

\renewcommand*{\bibfont}{\tiny}

\makeatletter
\setbeamertemplate{headline}{%
    \begin{beamercolorbox}[colsep=1.5pt]{upper separation line head}
    \end{beamercolorbox}
    \begin{beamercolorbox}{section in head/foot}
        \vskip2pt\insertnavigation{\paperwidth}\vskip2pt
    \end{beamercolorbox}%
    \begin{beamercolorbox}[colsep=1.5pt]{lower separation line head}
    \end{beamercolorbox}
}
\makeatother

\newcommand{\question}[1]{
    \begin{frame}[plain]
        \begin{columns}
            \column{.4\textwidth}
            \makebox[\columnwidth]{
                \includegraphics[width=\columnwidth,height=\paperheight,keepaspectratio]{mannetje.png}}
            \column{.6\textwidth}
            \large
            \textcolor{orange}{\textbf{\emph{#1}}}
        \end{columns}
    \end{frame}}

\newcommand{\instruction}[1]{\emph{\textcolor{gray}{[#1]}}}
