This course is set-up in the following manner. In a regular week, on the Mondays, we have lectures. Here, key concepts are explained from a conceptual and theoretical perspective, and examples of code implementations is python are provided.  On the Tuesdays, we have lab sessions. During the lab sessions, we generally start with knowledge questions about that week's literature as well as the preceeding lecture. Afterwards, students will work on assignments that are provided. When students have finished the assignments, they can work on the group assignment. The tuturial lecture (Isa) will walk around, and will help students with issues they encounter). If multiple students are encountering the same issues, these problems will be discussed plenarily. Active participation in the tutorial meetings is needed to ensure that students understand the materials, and can work on the (group) assignments.

\section*{Week 1: Course introduction \& Working with textual data}

\subsection*{Monday, 1--4:  Second Eastern Day}

In the first week of the course, we will not have a lecture on site due to Second Eastern day. However, we will provide an online lecture to kickstart the course. We also take a first look into using computational methods for communication science by discussing what we can learn from analyzing texts. We will discuss Bag-of-Words (BOW) representations of textual data, and discuss multiple ways of transforming text into matrices. 

You are encouraged to watch this online lecture  before the first tutorial lecture on Tuesday 2--4, or at your own convenience in the first week of the course. 


\subsection*{Tuesday, 2--4. Lab session}

textsc{\ding{52} Watch online lecture 1 (preferably before the first tutorial meeting)\\
\textsc{\ding{52} Read this manual and inspect the course Canvas page.}\\
\textsc{\ding{52} Make sure your computer is ready to use for the course }\\
\textsc{\ding{52} Read in advance: \cite{Hirschenberg2015}.} \\
\textsc{\ding{52}Read in advance: chapter 10: Text as data \cite{van_atteveldt_computational_2022}.} \\
\textsc{\ding{52}Read in advance: \cite{Boumans2016}.} \\

During the first tutorial meeting, we will introduce the course: We will explain the course goals and policies. 

We will take what we discussed in the lecture and use this to analyze text. We will start slow, and there will be time to go over some of the basic concepts as discussed in CCS-1. We will practice with some simple top-down and bottom-up algorithm for text analysis. In order to prepare ourselves for more advanced bottom-up techniques in week 2, we will start practicing with transforming textual data to matrices, using  \texttt{sklearn}'s vectorizers. 

In addition, you will form small groups of 3-4 students for the group assignment. 

\section*{Week 2:  Bottom up approaches to text analysis}

\subsection*{Monday, 8--4: Lecture}

\textsc{\ding{52}Read in advance chapter 11: Automatic analysis of text \cite{van_atteveldt_computational_2022}.} \\
\textsc{\ding{52}Read in advance: \cite{Brinberg2021}.} \\

In this meeting, we will discuss some prominent techniques to analyses text using bottom up techniques. Specifically, we will discuss latent dirichlet allocation, a populair approach to detect topics in textual data. In addition, we will focus on different ways to measure similarity between texts, using cosine and soft cosine similarity. 

\subsection*{Tuesdag, 9--4: Lab session}
During this lab session, you will complete the first set of MC-questions. 

During this tutorial meeting, we will practice with the concepts and code as introduced in Monday's lecture. We will try to measure the similarity between sets of documents using different approaches. In addition, we will practice with a simple LDA model. \\

In case you understand all concepts, and finished the in-class assignments, you can start working on the group assignment.  Specifically, we can start with exploring the dataset as provided, inspect relevant variables, and start working on cleaning the dataset and division of tasks. 

\section*{Week 3: Recommender systems}

\subsection*{Monday, 15--4: Lecture}
\textsc{\ding{52} Read in advance: \cite{Moller2018}.}\\
\textsc{\ding{52} Read in advance: \cite{Loecherbach2020}.}\\

In today's lecture, we will discuss different types of recommender systems. In order to understand how recommender systems work, an understanding of the concepts as discussed in previous weeks is crucial. Specifically, in order to build a recommender system, one needs to understand how to preprocess data, transfrom textual data to data matrices, and calculate similarity between texts. 

\subsection*{Tuesdag, 16--4: Lab session}
During this week's lab session, we will practice with designing a recommender system yourself. You will experiment with different settings, and inspect the effects thereof on the recommendation content. 

\section*{Week 5: Setting up supervised machine learning}

\subsection*{Monday, 22--4: Lecture}
\textsc{\ding{52} Watch: \url{https://www.youtube.com/watch?v=81vTqTz2pbM}.}\\
\textsc{\ding{52} Read in advance \cite{van_zoonen_social_2016}.}\\

This lecture is an introduction to supervised machine learning. We will discuss the basic principles of supervised machine learning using practical examples as well as example code. We will also discuss the role and applications of supervised machine learning in research and society.

\subsection*{Tuesday, 23--4: Lab session}
During this week's lab session, you will have the opportunity to get feedback on the group assignment and ask questions to your tutorial lecture. 

\section*{Week 6: Taking a break}

\subsection*{Monday, 29--4: No Lecture -- UvA teaching free week}
\subsection*{Tuesday, 30--4: No Lab session -- UvA teaching free week}

\textsc{\ding{52} Take a break}\\

\emph{Note that this week is an "education-free week", meaning that there will not be a lecture or a tutorial this week. Take a well-deserved brake and we continue the course in week 6!}

\section*{Week 7: Validation (in supervised machine learning)}

\subsection*{Deadline group project: Monday 6--5}
The deadline for handing in the group assignment is at the start of this week, \textbf{Monday 6--5 at 11:00AM}.

\subsection*{Monday, 6--5: Lecture}
\textsc{\ding{52} Read in advance \cite{jordan_mitchell}.} \\
\textsc{\ding{52} Read in advance \cite{meppelink_reliable_2021}.}\\

In this lecture, we will take a deeper dive into supervised machine learning whereby we also discuss how to validate classifiers.

\subsection*{Tuesday, 7--5: Lab session}
This week's lab session is a hands-on approached to supervised machine learning: you will set-up your own machine and train it to do some classifying. In addition, you will go through some approaches to validate your classifier.

\section*{Week 8: Coding in an academic context}

\subsection*{Monday, 23--5: Lecture}
\textsc{\ding{52} Read in advance \cite{baden_three_2022}.} \\

In the last lecture, we reflect on what we learned but mostly on how we can use this responsibly as scholars and researchers. 

\subsection*{Tuesday, 14--5: Lecture}
In the last lab session of the course, we will discuss scholars' use of code for research and the opportunities and responsibilities that this brings. It will also include a hands-on session on how to create responsible code.

\section*{Week 9: Exam week}

\subsection*{Tuesday, 21--5: Exam}
In the last lab session of the course, the exam will take place. 









